\title{Nature Neuroscience Submission Letter}
%
% See http://texblog.org/2013/11/11/latexs-alternative-letter-class-newlfm/
% and http://www.ctan.org/tex-archive/macros/latex/contrib/newlfm
% for more information.
%
\documentclass[11pt,stdletter,orderfromtodate,sigleft]{newlfm}
\usepackage{blindtext, xfrac, animate, hyperref, pxfonts}

  \setlength{\voffset}{0in}

\newlfmP{dateskipbefore=0pt}
\newlfmP{sigsize=20pt}
\newlfmP{sigskipbefore=10pt}
 
\newlfmP{Headlinewd=0pt,Footlinewd=0pt}
 
\namefrom{Jeremy R. Manning, Ph.D.}
\addrfrom{
	Dartmouth College\\
    Department of Psychological \& Brain Sciences\\
    HB 6207 Moore Hall\\
	Hanover, NH  03755}
 
\addrto{}
\dateset{\today}
 
\greetto{To the editors of \textit{Nature Neuroscience}:}
 
\closeline{Sincerely,}

\begin{document}
\begin{newlfm}
  We have enclosed our manuscript entitled \textit{High-level
    cognition during story listening is reflected in high-order
    dynamic correlations in neural activity patterns}, which we wish
  to submit for publication as a Brief Communication in \textit{Nature
    Neuroscience}.

  Our manuscript reports on a series of analyses carried out on
  neuroimaging data collected by Uri Hasson's group (Simony et al,
  2016) as participants listened to either an auditory story or a
  temporally scrambled version of the story.  We applied to this
  dataset a new computational framework for estimating high-order
  dynamic correlations that reflect ongoing cognitive processing.

  Prior work has indicated that dynamic functional (first-order)
  correlations---i.e., pairwise correlations between brain structures
  that change from moment to moment---change according to ongoing
  cognitive processes.  Our computational framework also enabled us to
  examine higher-order dynamic correlations.  For example,
  second-order correlations reflect \textit{homologous} patterns of
  correlation.  In other words, if the changing patterns of
  correlations between two regions, $A$ and $B$, are similar to those
  between two other regions, $C$ and $D$, this would be reflected in
  the second-order correlations between ($A$--$B$) and ($C$--$D$).  In
  this way, second-order correlations identify similarities and
  differences between subgraphs of the brain's connectome.
  Analogously, third-order correlations reflect homologies between
  second-order correlations-- i.e., homologous patterns of homologous
  interactions between brain regions.  More generally, higher-order
  correlations reflect homologies between patterns of lower-order
  correlations.  We can then ask: which ``orders'' of interaction are
  most reflective of high-level cognitive processes?

  When participants listened to an intact recording of the story, they
  exhibited similar (across-participants) high-order brain network
  dynamics.  These across-participant similarities are a reflection of
  the extent to which neural patterns are stimulus-driven, since the
  shared experimental stimulus was the only constant shared across
  participants.  By contrast, when participants instead listened to
  temporally scrambled recordings of the story, only lower-order brain
  network dynamics were similar across participants.  Further, the
  brain areas that mediate high-order correlations are associated with
  higher-level cognitive processes (cross-modality sensory
  integration, cognitive control, etc.), whereas low-order
  correlations and non-correlational activity dynamics most involve
  brain areas associated with lower-level cognitive processing
  (auditory processing, speech processing areas).  Our results
  indicate that higher orders of network interactions support
  higher-level aspects of cognitive processing

  In addition to the theoretical advances we report, identifying
  higher-order network dynamics associated with high-level cognition
  also required several important methods advances.  First, we used
  kernel-based dynamic correlations to extended the notion of (static)
  inter-subject functional connectivity (Simony et al., 2016) to a
  dynamic measure of inter-subject functional connectivity that does
  not rely on temporal sliding windows, and that may be computed at
  individual timepoints.  This allowed us to precisely characterize
  stimulus-evoked network dynamics that were similar across
  individuals.  Second, we developed a computational framework for
  efficiently and scalably estimating high-order dynamic correlations.
  Our approach uses dimensionality reduction algorithms and graph
  measures to obtain low-dimensional embeddings of patterns of network
  dynamics.  Third, we developed an analysis framework for identifying
  robust decoding results by carrying out our analyses using a range
  of parameter values and then identifying which results were robust
  to specific parameter choices.

  We have published and documented all of the data and analysis code
  used to generate the figures in our manuscript (link:
  \href{https://github.com/ContextLab/timecorr-paper}{github.com/ContextLab/timecorr-paper}),
  along with a concurrent release of a Python toolbox for carrying out
  analogous analyses on new datasets (link:
  \href{http://timecorr.readthedocs.io}{timecorr.readthedocs.io}).

  We have suggested the following scientists as expert reviewers:
  Richard Betzel (rbetzel@indiana.edu), Danielle Bassett (dsb@seas.upenn.edu), Emily Finn
  (emily.finn@nih.gov), Olaf Sporns (osporns@indiana.edu), and Christopher Honey (chris.honey@jhu.edu).

  We note that we have included the complete methods section in the
  main text to facilitate peer review.  If deemed potentially suitable
  for publication, we will update our manuscript by moving the current
  methods section into our supporting document, and by adding a
  methods summary conforming to the Brief Communications format.
  Thank you for considering our manuscript for publication in Nature
  Neuroscience.

\end{newlfm}
\end{document}


