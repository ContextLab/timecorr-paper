\documentclass[english]{article}
%\usepackage[utf8]{inputenc}
\usepackage{subfig}
\usepackage{graphicx}
\usepackage{subfig}
\usepackage{babel,blindtext}
\usepackage{amsmath}
\usepackage{hyperref}
\usepackage{setspace}
\usepackage{apacite}
\usepackage{fancyhdr}
\pagestyle{fancy}
\usepackage{natbib}
\usepackage{pxfonts}
\usepackage[utf8]{inputenc}
\usepackage{xfrac}
\fancyhf{}
\renewcommand{\headrulewidth}{0pt}
\fancyhead[R]{\thepage}
\usepackage{multirow,booktabs,setspace,caption}
\usepackage{tikz}
\captionsetup[figure]{labelsep=period,labelfont=it,justification=justified,
  singlelinecheck=false,font=doublespacing}
  \usepackage[left=1in,right=1in,top=1in,bottom=1in]{geometry}
\lhead{}
\fancyheadoffset[R]{0cm}
\cfoot{}

\fancypagestyle{first page}{
  \lhead{Running head: High-order dynamic correlations}
  \rhead{\thepage}
}

\begin{document}
\title{A scalable approach to inferring the high-order dynamic correlations underlying multi-dimensional timeseries data}
\author{Lucy L. W. Owen$^1$,
Paxton C. Fitzpatrick$^1$,
Kirsten Ziman$^1$,\\
Andrew C. Heusser$^1$,
Stephen F. Satterthwaite$^1$,
Thomas Hao Chang$^{1,2}$,
and\
Jeremy R. Manning\textsuperscript{$1, \dagger$}\\
[0.1in]$^1$Department of Psychological and Brain Sciences, Dartmouth College, Hanover, NH\\
$^2$Amazon.com, Seattle, WA\\
\textsuperscript{$\dagger$}Address correspondence to jeremy.r.manning@dartmouth.edu}

\begin{titlepage}
\thispagestyle{fancy}   
% \center
% \vspace*{\fill}
% \huge{}\\[0.5cm]

% \Large{author 1,
% author 2,
% author 3, and
% Jeremy R. Manning\textsuperscript{$\dagger$}}\\[0.5cm]

% \large{Department of Psychological and Brain Sciences, Dartmouth College, Hanover, NH 03755}\\[0.5cm]

% \textsuperscript{$\dagger$} Address correspondence to jeremy.r.manning@dartmouth.edu

% \vspace*{\fill}
\end{titlepage}



% \addtocounter{page}{1}
% \rhead{\thepage}

\maketitle


\begin{abstract}
Write an abstract...

\end{abstract}

\doublespacing

\section*{Introduction}
The dynamics of the observable universe are meaningful in three respects.  First, the behaviors of the \textit{atomic units} that exhibit those dynamics are highly interrelated.  The actions of one unit typically have implications for one or more other units.  In other words, there is non-trivial \textit{correlational structure} defining how different units interact with and relate to each other.  Second, that correlational structure is \textit{hierarchical} in the sense that it exists on many spatiotemporal scales.  The way one group of units interacts may relate to how another group of units interact, and the interactions between those groups may exhibit some rich structure.  Third, the structure at each level of this correlational hierarchy changes from moment to moment, reflecting the ``behavior'' of the full system.

These three properties (rich correlations, hierarchical organization, and dynamics) are major hallmarks of many complex systems.  For example, within a single cell, the cellular components interact at many spatiotemporal scales, and those interactions change according to what that single cell is doing.  Within a single human brain, the individual neurons interact within each brain structure, and the structures interact to form complex networks.  The interactions at each scale vary according to the functions our brains are carrying out.  And within social groups, interactions at different scales (e.g. between individuals, family units, communities, etc.) vary over time according to changing goals and external constraints.

Although many systems exhibit rich dynamic correlations at many scales, a major challenge to studying such patterns is that typically neither the correlations nor the hierarchical organizations of those correlations can be directly observed.  Rather, these fundamental properties must be inferred indirectly by examining the observable parts of the system-- e.g. the behaviors of the individual atomic units of that system.  Here we propose a series of mathematical operations that may be used to recover dynamic correlations at a range of scales.

There are two basic steps to our approach, which we describe in detail in the \textit{Methods} section.  In the first step, we take a number-of-timepoints ($T$) by number-of-features ($F$) \textit{matrix of observations} ($\mathbf{X}$) and we return a $T$ by $\frac{F^2 - F}{2}$ \textit{matrix of dynamic correlations} ($\mathbf{Y}$).  Here $\mathbf{Y_0}$ describes, at each moment, how all of the features (columns of $\mathbf{X}$) are inferred to be interacting.  (Since the interactions are assumed to be non-recurrent and symmetric, only the upper triangle of the full correlation matrix is computed.)  In the second step, we use a dimensionality reduction technique to project $\mathbf{Y_0}$ onto an $F$ dimensional space, resulting in a new $T$ by $F$ matrix $\mathbf{Y_1}$  Note that $\mathbf{Y_1}$ contains information about the correlation dynamics present in $\mathbf{X}$, but represented in a compressed number of dimensions.  By repeatedly applying these two steps in sequence, we can examine and explore higher order dynamic correlations in $\mathbf{X}$.  In the \textit{Results} section we demonstrate how our approach may be applied to multi-dimensional timeseries data in several domains: human brain data, financial data, social network data, and climate data. 

\section*{Methods}

\begin{align}
\mathrm{corr}_t(x,y) &= \frac{\sum_{t=1}^T \left( x_t - \bar{x_t} \right) \left( y_t - \bar{y_t} \right)}{\sqrt{\sum_{t=1}^{T}\sigma ^{2}_{x_{t}}\sigma ^{2}_{y_{t}}}}\mathrm{, where}\\
\bar{a_t} &= \sum_{t=1}^T w(t)_i a_i\\
\sigma^{2}_{a_t} &= \sum_{t=1}^{T} \left(a_t - \bar{a_t} \right)^2\\
w(t) &= \mathcal{N}\left(1...T ~|~ t, \sigma_\mathcal{N}\right) %update to include eye, laplace, and gaussian weights
\end{align}



% \begin{equation}
% c(t) = \sum_{i=1}^{T}w(t)_{i}
% \end{equation}



\section*{Results}
%currently Lucy, Kirsten, and Paxton have agreed to make figures

%%%%%%%%%%%%%%%%%%%%%%%%%%%%%%%%%%%%%%%
\section*{Discussion}

\subsection*{Concluding remarks}

\section*{Acknowledgements}
Our work was supported in part by DARPA RAN grant number XXXX and NSF EPSCoR Grant Number XXXX.  We acknowledge useful discussions with Luke Chang, Hany Farid, Qiang Liu, Matthijs van der Meer, and Emily Whitaker.  The content is solely the responsibility of the authors and does not necessarily represent the official views of our supporting organizations.

\bibliographystyle{apacite}
\bibliography{memlab}

\end{document}

